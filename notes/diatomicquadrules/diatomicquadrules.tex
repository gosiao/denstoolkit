\documentclass[11pt]{amsart}
\usepackage{geometry}                % See geometry.pdf to learn the layout options. There are lots.
\geometry{letterpaper}                   % ... or a4paper or a5paper or ... 
%\geometry{landscape}                % Activate for for rotated page geometry
%\usepackage[parfill]{parskip}    % Activate to begin paragraphs with an empty line rather than an indent
\usepackage{amsmath}
\usepackage{graphicx}
\usepackage{amssymb}
\usepackage{epstopdf}
\DeclareGraphicsRule{.tif}{png}{.png}{`convert #1 `dirname #1`/`basename #1 .tif`.png}
\usepackage{color}


\newcommand{\vecx}{\boldsymbol{x}}
\newcommand{\tnabla}{\tilde{\nabla}}
\newcommand{\ds}{\partial_{s}}
\newcommand{\us}{\hat{s}}
\newcommand{\dt}{\partial_{t}}
\newcommand{\ut}{\hat{t}}
\newcommand{\dA}{\dot{A}}
\newcommand{\dB}{\dot{B}}


\title{Diatomic cubature rules}
\author{J. M. Solano-Altamirano}
\date{\today}                                           % Activate to display a given date or no date

\begin{document}
\begin{abstract}
We describe how to construct cubature rules for diatomic molecules. In the current version,
atoms are required to be located along the $z$-axis for the rules to work properly.
\end{abstract}
\maketitle


%++++++++++++++++++++++++++++++++++++++++++++++++++++++++++++++++
\section{Core cubature}
%++++++++++++++++++++++++++++++++++++++++++++++++++++++++++++++++
%----------------------------------------------------------------
\subsection{Upper domain}
%----------------------------------------------------------------

If the integration domain is an off-centred sphere of radius $a$, whose centre is displaced along
the global $z$-axis, denoted here as $\Sigma_0$, then
the integral of a azimuthally symmetric function can be reduced to twice the integral
over the two dimensional domain $\Gamma_0$, defined as (see Fig.~\ref{fig:offcentsphdom}
for its geometry):
\begin{equation}
	\Gamma_0\equiv\left\{
		(\rho,\phi) |a\leq\rho\leq0,0\leq\phi\leq\pi
	\right\}.
\end{equation}
%
\textit{i.e.,}
\begin{eqnarray}
	\int_{\Sigma_0}f(r,\theta,\varphi)dV=2\pi\int_0^{\pi}\sin\phi d\phi\int_0^a%
	\rho^2f(\rho,\phi)d\rho\equiv2\pi I.
\end{eqnarray}

\begin{figure}[htb!]
   \centering
   \includegraphics[width=0.4\textwidth]{images/numIntegDiatCore}
   \caption{Scheme of a positively $z$-translated spherical domain, relative to the origin.
   The domain is rotated and translated in such a manner that the spherical domain is
   aligned along the $z$-axis. The scheme shows the projection of the system onto the global 
   $x$-$z$ plane. $a$ is the radius of the spherical domain and $d_u$ is the distance
   between the upper domain centre and the origin of coordinates. The variables
   $(r,\theta)$ and $(\rho,\phi)$ are the polar and radial and polar angles
   relative to the local (primed) and global (non-primed) coordinate systems, respectively.}
   \label{fig:offcentsphdom}
\end{figure}

The numerical scheme (cubature) can be build applying the following change of variables:
\begin{equation}
	p=\frac{2\rho}{a}-1
\end{equation}
and
\begin{equation}
	q=\cos\phi.
\end{equation}
After applying the changes of variables we get:
\begin{equation}
	I=\int_{-1}^1dq\int_{-1}^1\frac{a^3}{8}(p+1)^2f(p,q).
\end{equation}

The cubature rules can be constructed using standard Gauss-Legendre quadrature rules for
$p$ and $q$, \textit{i.e.}, the cubature rules for $I$ are:
%
\begin{equation}
	I=\sum_{i,j}W_iw_jf(p_i,q_j),
\end{equation}
%
where
%
\begin{equation}
	W_i\equiv\left(w_i\frac{a^3}{8}(p_i+1)^2\right),
\end{equation}
%
and $w_i$ ($p_i$) and $w_j$ ($q_j$) are the standard Gauss-Legendre weights (abscissas) related to
$p$ and $q$, respectively.
%
The local variables are given by:
%
\begin{subequations}
\begin{eqnarray}
	\rho_{i} & = & \frac{a}{2}(p_i+1),\\
	\phi_j & = & \arccos q_j.
\end{eqnarray}
%
\end{subequations}

Finally, the global spherical coordinates are given by:
%
\begin{subequations}
\begin{eqnarray}
	r^2_{ij} & = & \frac{a^2}{4}(p_i+1)^2+a(p_i+1)q_jd_u+d_u^2,\\
	\varphi_{ij} & = & 0,\\
	\theta_{ij} & = & \arcsin\left(\frac{a(p_i+1)}{2r_{ij}}\sqrt{1-q_j^2}\right),
\end{eqnarray}
\end{subequations}
%
and the global Cartesian coordinates are:
%
\begin{subequations}
\begin{eqnarray}
	x_{ij} & = & \frac{a}{2}(p_i+1)\sqrt{1-q_j^2},\\
	y_{ij} & = & 0,\\
	z_{ij} & = & \frac{a}{2}(p_i+1)q_j+d_u.
\end{eqnarray}
%
\end{subequations}

%----------------------------------------------------------------
\subsection{Lower domain}
%----------------------------------------------------------------

The lower cubature rules are obtained by setting $d_u\rightarrow -d_l$,
$a\rightarrow b$, and $z\rightarrow-z$. Here $b$ is the radius of the lower
spherical region of integration, and the centre of it is located at $z=-d_l$.
The weights are the same as for the upper domain, the global spherical coordinates
are:
%
\begin{subequations}
\begin{eqnarray}
	r^2_{ij} & = & \frac{b^2}{4}(p_i+1)^2-b(p_i+1)q_jd_l+d_l^2,\\
	\varphi_{ij} & = & 0,\\
	\theta_{ij} & = & \arcsin\left(\frac{b(p_i+1)}{2r_{ij}}\sqrt{1-q_j^2}\right),
\end{eqnarray}
\end{subequations}
%
and the global Cartesian coordinates are:
%
\begin{subequations}
\begin{eqnarray}
	x_{ij} & = & \frac{b}{2}(p_i+1)\sqrt{1-q_j^2},\\
	y_{ij} & = & 0,\\
	z_{ij} & = & -\frac{b}{2}(p_i+1)q_j-d_l.
\end{eqnarray}
%
\end{subequations}

%++++++++++++++++++++++++++++++++++++++++++++++++++++++++++++++++
\section{Valence cubature}
%++++++++++++++++++++++++++++++++++++++++++++++++++++++++++++++++
%----------------------------------------------------------------
\subsection{Upper domain}
%----------------------------------------------------------------
For the valence integral, \textit{i.e.}, the integral over the region comprised of an off-centred
spherical domain with two cavities, denoted as $\Sigma_{2c}$,
we follow the setup of the cubature for a spherical domain with two cavities with slight modifications.
%
\begin{figure}[htb!]
   \centering
   \includegraphics[width=0.4\textwidth]{images/numIntegDiatValence}
   \caption{Scheme of a positively $z$-translated spherical domain, $\Gamma_{2c}$,
   relative to the origin, with two spherical cavities.
   The domain is rotated and translated in such a manner that $\Gamma_{2c}$ is
   aligned along the $z$-axis. The scheme shows the projection of $\Gamma_{2c}$ onto the global 
   $x$-$z$ plane. $a$ is the radius of the spherical cavity, $b$ is the radius of $\Gamma_{2c}$,
   $c$ is the $z$-coordinate of the
   plane that cuts $\Gamma_{2c}$ in two halves, $d$ is the $z$-coordinate of the cavity centre,
   and $d_u=d-c$ is the distance
   between the upper cavity's centre and the cutting plane. The variables
   $(r,\theta)$ and $(\rho,\phi)$ are the polar and radial and polar angles
   relative to the local (primed) and global (non-primed) coordinate systems, respectively.}
   \label{fig:offcentsphdom2cav}
\end{figure}
%
Hence, the integral of an azymuthally symmetric function $f(r,\theta)$ over $\Sigma_{2c}$ can be reduced
to two integrals, each of which is an integral over the shadowed region depicted in Fig.~\ref{fig:offcentsphdom2cav}, which is
defined as:
%
\begin{eqnarray}
	\Gamma_{2c} & \equiv & \Big\{(\rho,\phi)|a\leq\rho\leq\sqrt{b^2-d^2\sin^2\phi}-d\cos\phi,
	\ 0\leq\phi\leq\phi_1\Big\}\\\nonumber
	 & & \bigcup\Big\{ (\rho,\phi) | a\leq\rho\leq-d_u\sec\phi,%
	 \phi_1\leq\phi\leq\pi \Big\}.
\end{eqnarray}
%
The integral over the upper half ($\Sigma_{2c}^{u}$) is given by:
%
\begin{eqnarray}
	\int_{\Sigma_{2c}^{u}}f(r,\theta)dV & = & \phantom{+}2\pi\int_0^{\phi_1}\sin\phi d\phi
		\int_a^{\sqrt{b^2-d_u^2\sin^2\phi}-d_u\cos\phi}\rho^2f(\rho,\phi)d\rho\\\nonumber
		 & & +2\pi\int_{\phi_1}^{\pi}\sin\phi d\phi\int_{a}^{-d_u/\cos\phi}\rho^2f(\rho,\phi)d\rho\\\nonumber
		 & & \equiv 2\pi(M_u+N_u)
\end{eqnarray}
%
The angle $\phi_1$ is:
%
\begin{equation}
	\phi_1=\pi-\arctan\left(\frac{b}{d_u}\right).
\end{equation}
%

%................................................................
\subsubsection{$M_u$ integral}
%................................................................
The cubature for the $M_u$ integral is achieved through the changes of variables:
%
\begin{subequations}\label{eq:chgvarMGamm2c}
\begin{eqnarray}
	\nu & \equiv & \cos\phi\\
	p & \equiv & \frac{2\rho+\nu d_u-a-\sqrt{b^2-d_u^2(1-\nu^2)}}%
		{\sqrt{b^2-d_u^2(1-\nu^2)}-\nu d_u-a},\\
	q & \equiv & \frac{2\nu-1-\nu_1}{1-\nu_1}.
\end{eqnarray}
\end{subequations}
%

Notice that $d_u=d-c$, and $c$ might be positive or negative.
After substituting Eqs.~(\ref{eq:chgvarMGamm2c}) into $M$, we obtain:
%
\begin{equation}
	M_u=\frac{1-\nu_1}{4}\int_{-1}^1dq\int_{-1}^1\left(\sqrt{b^2-d_u^2(1-\nu^2)}-\nu d_u-a\right)
		\rho^2(p,q)f(\rho(p,q),\nu(q))dp.
\end{equation}
%
Thus, the cubature rules (using $\{w_i,p_i\}$ and $\{w_j,q_j\}$ as weights and abscissas) are:
%
\begin{equation}
	M_u=\sum_{i,j}W_{ij}
		\rho^2_{ij}f(\rho_{ij},\nu_j).
\end{equation}
%
Here:
%
\begin{subequations}\label{eq:nujrhoijMupDefs}
\begin{eqnarray}
	W_{ij} & \equiv & \frac{1-\nu_1}{4}%
	\left(\sqrt{b^2-d_u^2(1-\nu_j^2)}-\nu_jd_u-a\right)w_iw_j\\
	\nu_j & \equiv & \frac{1-\nu_1}{2}q_j+\frac{\nu_1+1}{2},\\
	\rho_{ij} & \equiv & \left(\frac{\sqrt{b^2-d_u^2(1-\nu_j^2)}-\nu_jd_u-a}{2}\right)p_i
	             + \frac{\sqrt{b^2-d_u^2(1-\nu_j^2)}-\nu_jd_u+a}{2}.
\end{eqnarray}
\end{subequations}
%
The global Cartesian coordinates are given by:
%
\begin{subequations}
\begin{eqnarray}
	x_{ij} & = & \rho_{ij}\sqrt{1-\nu_j^2},\\
	y_{ij}& = & 0,\\
	z_{ij} & = & \rho_{ij}\nu_j+d_u+c=\rho_{ij}\nu_j+d.
\end{eqnarray}
\end{subequations}
%
Notice that $c$ might be positive or negative.

The global spherical coordinates are then found as:
%
\begin{subequations}
\begin{eqnarray}
	r_{ij}^2 & = & x_{ij}^2+z_{ij}^2=\rho_{ij}^2+2\nu_j\rho_{ij}d+d^2,\\
	\theta_{ij}& = & \arcsin\left(\frac{\rho_{ij}}{r_{ij}}\sqrt{1-\nu_j^2}\right),\\
	\phi_{ij} & = & 0.
\end{eqnarray}
\end{subequations}
%

%................................................................
\subsubsection{$N_u$ integral}
%................................................................
To determine the cubature rules for $N_u$, we use the following change of variables:
%
\begin{subequations}\label{eq:chgvarNGamm2c}
\begin{eqnarray}
	\nu & \equiv & \cos\phi\\
	s & \equiv & \frac{-d_u/\nu}{d_u/\nu+a},\\
	t & \equiv & \frac{2\nu+1-\nu_1}{1+\nu_1}.
\end{eqnarray}
\end{subequations}
%

Substituting Eqs.~(\ref{eq:chgvarNGamm2c}) into $N$ renders:
%
\begin{equation}
	N_u=\frac{1+\nu_1}{4}\int_{-1}^1dt\int_{-1}^1\left(\frac{-2d_u}%
		{(\nu_1+1)t+\nu_1-1}-a\right)
		\rho^2(s,t)f(\rho(s,t),\nu(t))ds.
\end{equation}
%
Thus, the cubature rules (using $\{w_i,s_i\}$ and $\{w_j,t_j\}$ as weights and abscissas) are:
%
\begin{equation}
	N_u=\sum_{i,j}W_{ij}
		\rho^2_{ij}f(\rho_{ij},\nu_j).
\end{equation}
%
Here,
%
\begin{subequations}\label{eq:nujrhoijNupDefs}
\begin{eqnarray}
	W_{ij} & \equiv & \frac{1+\nu_1}{4}%
	\left(\frac{-2d_u}{(\nu_1+1)t_j+\nu_1-1}-a\right)w_iw_j\\
	\nu_j & \equiv & \frac{1+\nu_1}{2}t_j+\frac{\nu_1-1}{2},\\
	\rho_{ij} & \equiv & \left(\frac{-a-d_u/\nu_j}{2}\right)s_i
	             + \frac{a-d_u/\nu_j}{2}.
\end{eqnarray}
\end{subequations}
%
The global Cartesian coordinates are given by:
%
\begin{subequations}
\begin{eqnarray}
	x_{ij} & = & \rho_{ij}\sqrt{1-\nu_j^2},\\
	y_{ij}& = & 0,\\
	z_{ij} & = & \rho_{ij}\nu_j+d_u+c=\rho_{ij}\nu_j+d.
\end{eqnarray}
\end{subequations}
%
Notice again that $c$ might be positive or negative.

The global spherical coordinates are then found as:
%
\begin{subequations}
\begin{eqnarray}
	r_{ij}^2 & = & x_{ij}^2+z_{ij}^2=\rho_{ij}^2+2\nu_j\rho_{ij}d+d^2,\\
	\theta_{ij}& = & \arcsin\left(\frac{\rho_{ij}}{r_{ij}}\sqrt{1-\nu_j^2}\right),\\
	\phi_{ij} & = & 0.
\end{eqnarray}
\end{subequations}
%

%----------------------------------------------------------------
\subsection{Lower domain}
%----------------------------------------------------------------

The lower cubature rules are obtained by setting $d_l=-d+c$,
$a\rightarrow a_l$, and $z\rightarrow-z$. Here $a_l$ is the radius of the lower
spherical region of integration, and its centre is located at $z=d=-|d|$.
The weights are the same as for the upper domain.

%................................................................
\subsubsection{$M_l$ and $N_l$ integrals}
%................................................................
%
%
The global Cartesian coordinates are given by:
%
\begin{subequations}
\begin{eqnarray}
	x_{ij} & = & \rho_{ij}\sqrt{1-\nu_j^2},\\
	y_{ij}& = & 0,\\
	z_{ij} & = & -\rho_{ij}\nu_j-d_l+c=-\rho_{ij}\nu_j+d=-\rho_{ij}\nu_j-|d|.
\end{eqnarray}
\end{subequations}
%
Notice that $c$ might be positive or negative, and here $d<0$.

The global spherical coordinates are then found as:
%
\begin{subequations}
\begin{eqnarray}
	r_{ij}^2 & = & x_{ij}^2+z_{ij}^2=\rho_{ij}^2+2\nu_j\rho_{ij}|d|+d^2,\\
	\theta_{ij}& = & \arcsin\left(\frac{\rho_{ij}}{r_{ij}}\sqrt{1-\nu_j^2}\right),\\
	\phi_{ij} & = & 0.
\end{eqnarray}
\end{subequations}
%

Substitute Eqs.~(\ref{eq:nujrhoijMupDefs}) or (\ref{eq:nujrhoijNupDefs}) as needed for
the $M_l$ or  $N_l$ integral, respectively.


\end{document}  
