Usage:

	dtkline wf?name [option [value(s)]] ... [option [value(s)]]

Where wf?name is the input wfx(wfn) name, and options can be:
Where wf?name is the input wfx(wfn) name, and options can be:

  -a a1 a2  	Define the atoms  (a1,a2) used to define the line.
            	  If this option is not activated, the program will 
            	  define the line using the first atom and the vector
            	  (1,1,1).
  -n  dim   	Set the number of points for the dat file
  -o outname	Set the output file name.
            	  (If not given the program will create one out of
            	  the input name; if given, the gnp file and the pdf will
            	  use this name as well --but different extension--).
  -p prop	Choose the property to be computed. prop is a character,
         	  which can be (d is the default value): 
         		d (Density)
         		g (Magnitude of the Gradient of the Density)
         		l (Laplacian of density)
         		K (Kinetic Energy Density K)
         		G (Kinetic Energy Density G)
         		E (Electron Localization Function -ELF-)
         		L (Localized Orbital Locator -LOL-)
         		M (Magnitude of the Gradient of LOL)
         		P (Magnitude of the Localized Electrons Detector -LED-)
         		r (Region of Slower Electrons -RoSE-)
         		s (Reduced Density Gradient -s-)
         		S (Shannon Entropy Density)
         		u (Scalar Custom Field)
         		V (Molecular Electrostatic Potential)
         		v (Virial Potential Energy Density)
         		e (Ellipticity)
  -P     	Create a plot using gnuplot.
  -k     	Keeps the *.gnp file to be used later by gnuplot.
  -z     	Compress the cube file using gzip (which must be intalled
         	   in your system).
  -V        	Displays the version of this program.
  -h     	Display the help menu.

  --help    		Same as -h
  --version 		Same as -V
********************************************************************************
  Note that the following programs must be properly installed in your system:
                                    gnuplot
                                    epstopdf
                                      gzip
********************************************************************************
