%%                     This source code is part of
%% 
%%                   D  E  N  S  T  O  O  L  K  I  T
%% 
%%                          VERSION: 1.5.0
%% 
%%              Contributors: Juan Manuel Solano-Altamirano
%%                            Julio Manuel Hernandez-Perez
%%                            Luis Alfredo Nunez-Meneses
%%         Copyright (c) 2013-2021, Juan Manuel Solano-Altamirano
%%                                  <jmsolanoalt@gmail.com>
%% 
%%  -------------------------------------------------------------------
%%  Permission is granted to copy, distribute and/or modify this document
%%  under the terms of the GNU Free Documentation License, Version 1.3
%%  or any later version published by the Free Software Foundation;
%%  with no Invariant Sections, no Front-Cover Texts, and no Back-Cover Texts.
%%  A copy of the license is included in the section entitled "GNU
%%  Free Documentation License".
%%  ---------------------------------------------------------------------
%% 
%%  If you want to redistribute modifications of the suite, please
%%  consider to include your modifications in our official release.
%%  We will be pleased to consider the inclusion of your contributions.
%%  within the official distribution. Please keep in mind that
%%  scientific software is very special, and version control is 
%%  crucial for tracing bugs. If in despite of this you distribute
%%  your modified version, please do not call it DensToolKit.
%% 
%%  If you find DensToolKit useful, we humbly ask that you cite
%%  the paper(s) on the package --- you can find them on the top
%%  README file.
%% 
%% ************************************************************************
%%%%


\chapter{Introduction}

\section{About \DTK}

DensToolKit (DTK) is a suite of programs designed to analyze scalar
and vector fields related to the 
electron density of a molecule. The purpose of breaking down the computational tasks into a 
set of small programs, as opposed use a big single complex program, 
is having a group of programs that can be used in scripts. The philosophy behind this
approach was learned from Linux OS.

While producing a high quality plot, for scientific publishing purposes, requires a reasonable
amount of effort, spending time on doing plots to visualise intermediate results often results
in a waste of time. After using \texttt{gnuplot} for a couple of years, we realized that
it is more efficient to build small scripts exploiting the Linux OS tools' capabilities, and
the scripting capabilities of \texttt{gnuplot}. This, very frequently, reduces the effort and
the time invested in producing plots and reviewing them. In addition, by increasing the amount
of analysed data, one may find information that would had not been possible to find 
by analysing smaller amounts of data. 

On the hand, there is a considerable variety of programs written in fortran (especially in
fortran 77) in the computational chemistry field (here we are referring to the programs
designed to study the electron density and its derived fields). However, in order to maintain the
compatibility with fortran compilers, the programmers are not taking the advantage of one of
the greatest features of the object-oriented programming: the reusability of the software.
We aim to fill part of this gap by providing a seed code, which can be used to start an
open-source project. \DTK{} is written in \texttt{C++}, and it has some object-oriented 
design (we will keep working on improving the design).

Another important feature of \DTK{} is its portability. \DTK{} can be compiled in at least three of the major
operative systems, namely Linux/Unix, Darwin(Mac OSX) and Windows (cygwin), using the gnu-gcc compiler.
In addition, the pure numerical capabilities of \DTK{} do not require any library, nor any other
program to be installed. However, many of the programs are able to produce script files
which can be parsed to \texttt{gnuplot}, \texttt{VMD}, and \texttt{povray}. This method is chosen as a
medium to provide a simple method for visualising the data obtained by the suite, without
re-inventing the wheel: we cannot (and will not) compete with the amazing results that can 
be obtained with these programs (and others such as \texttt{GraphicsMagick, VMD,} etc.)

\section{Installation guide}

Here we describe how to install \DTK{} on a range of UNIX-like operating systems.

\DTK{} have been tested on Linux, MacOSX, and Microsoft Windows 7, and we believe it should
work under almost any POSIX system. The compilation on MS Windows 7 was performed under cygwin;
regrettably, full support for this operative system is not within our immediate plans.

\subsection{Prerequisites}

Every program of the suite \DTK{} is divided into two main parts. The most fundamental
(and the core of \DTK) is the calculation of scalar and vector fields related to the electron
density. Generally, the data obtained from the programs are saved into *.log, *.dat, *.tsv,
or *.cub files. 

Additionally, \DTK{} can perform calls to third party programs to generate some plots.
Most of the programs use \texttt{gnuplot}, but some times \texttt{povray} is also called.
As a general rule, the programs create a small script that can be passed directly to
\texttt{gnuplot}/\texttt{povray}. The idea is to facilitate the creation of
high-quality plots using those programs, by having a tool that easily and 
quickly creates plots
of what has been calculated, skipping the time-wasting use of graphical interfaces.
In addition, a few scripts are provided in order to further the automatisation of
rendering plots. These scripts uses third party programs such as \texttt{epstool},
\texttt{epstopdf}, etc.

That being said, before you compile \DTK{}, we recommend the installation of the following packages
in your system:
\begin{itemize}
   \item \textbf{\texttt{gnuplot} 4.6 or later}. Under Linux, you can simply type
   \begin{verbatim}
      #apt-get install gnuplot
      #yum install gnuplot
   \end{verbatim}
   (apt-get for Debian/Linux and its variants, yum for systems using RPMs), and under MacOSX
   \begin{verbatim}
      #port install gnuplot
   \end{verbatim}
   Here, it has been assumed that macports is already installed in your system, and the character \texttt{\#} means that you should run this commands as root/superuser.
   
Unless stated otherwise, all of the third-party programs can be installed using this same method
under Linux or MacOSX.

For MS Windows, you can get almost all the required programs by running the cygwin installer,
and choosing the packages from the list. \texttt{Gnuplot} and \texttt{povray} also have independent
installers on their websites.

More information about \texttt{gnuplot} can be found in: \url{http://www.gnuplot.info}

Below there is a list of the additional programs required to fully exploit the capabilities of \DTK.

   \item \textbf{\texttt{povray} 3.6 or later}: \url{http://www.povray.org}
   \item \textbf{ghostscript 9 or later}: \url{http://www.ghostscript.com}
   \item \textbf{epstool 3.08 or later}: \url{http://pages.cs.wisc.edu/~ghost/gsview/epstool.htm}
   \item \textbf{epstopdf 2.19 or later}: \url{http://www.ctan.org/pkg/epstopdf}
   \item \textbf{graphicsmagick 1.3.18 or later} \url{http://www.graphicsmagick.org}
   \item \textbf{gzip 1.6 or later} \url{http://www.gzip.org}
\end{itemize}

None of the listed programs are required to perform the calculations nor for generating *.log, *.dat, *.tsv, nor *.cub files. If you do not wish to install them (although we strongly recommend it), you can still run the programs and all the data files will be generated anyway (perhaps with several error or warning messages). Later you can generate plots using your preferred software.

\section{Installation from source archive}

In this section, we assume you have basic knowledge of how to install simple programs in Linux through the command line (the same applies to MacOSX and MS Windows ---cygwin---).

\begin{itemize}
\item After obtaining a copy of the source file, unpack the distributed *.tar.gz:
\begin{lstlisting}[language=bash]
$tar zxvf denstoolkit-YYYYMMDD-HHMM.tar.gz
$cd denstoolkit-YYYYMMDD-HHMM/src
\end{lstlisting}
\item Just to verify that you have all the required packages installed, run
\begin{lstlisting}[language=bash]
$./checkdependencies
\end{lstlisting}
 If a recommended package is missing, you will see a message. You can skip this step, however, if this simple script works, then the remaining of the installation should be easy.
 
 Especial attention must be paid to the \texttt{povray} program if you are using cygwin: After installing the \texttt{povray} binaries, look at the direct access created by the installer and (after right-clicking) check properties. In the properties window, you can see the full path of the program. You must add the path to your local \texttt{.bashrc} by adding at the end of this file the line
\begin{lstlisting}[language=bash]
PATH=$PATH:/directory/where/povray/is
\end{lstlisting}
After this is done, type from the command line
\begin{lstlisting}[language=bash]
$pvengine.exe
\end{lstlisting}
(if you use the new pvengine64.exe, then that is what you should type.) If \texttt{povray} opens, then you have configured it right.

After all the required programs have been installed, the script \texttt{checkdependencies} will display the message: \texttt{All required packages are installed, you can proceed to build and install \DTK!}
\item Set the correct command for \texttt{povray} (MS Windows only): Please, make yourself sure that you can call \texttt{povray} from the command line (if you have successfully run the script \texttt{checkdependencies}, see above, then the next step is straightforward). Modify the Makefile by changing the line
\begin{lstlisting}[language=make]
USERPOVCMD = povray
\end{lstlisting}
by
\begin{lstlisting}[language=bash]
USERPOVCMD = pvengine.exe
\end{lstlisting}
(or ``\texttt{USERPOVCMD = pvengine64.exe}'').
\item Build the \DTK{} binaries:
\begin{lstlisting}[language=bash]
$make
\end{lstlisting}
\item Install the binaries into the \texttt{INSTALL\_PATH} directory
  \begin{lstlisting}[language=bash]
$sudo make install
\end{lstlisting}
   By default, the final binaries of \DTK{} will be installed in \texttt{/usr/local/bin}.
   You can change this by defining the variable \texttt{INSTAPP\_PATH} when running make:
\begin{lstlisting}
$sudo make INSTALL_PATH=/the/custom/install/path install
\end{lstlisting}
\item Run some tests (optional).
\begin{lstlisting}[language=bash]
$make runtest
\end{lstlisting}
 You can review the sample output files in the directory\\
\texttt{denstoolkit-YYYYMMDD-HHMM/outputs}
\end{itemize}

\section{Uninstalling \DTK}

In the source directory \texttt{denstoolkit-YYYYMMDD-HHMM/src} type:
\begin{lstlisting}[language=bash]
$sudo make fullclean
$sudo rm /usr/local/bin/dtk*
\end{lstlisting}
Change \texttt{/usr/local/bin} by the install path you used during installation.
This will remove all the intermediate files used during compilation, all the binaries, and all the outputs produced (if you ran \texttt{\$make runtest}). After this, you can safely remove the entire directory \texttt{denstoolkit-YYYYMMDD-HHMM}.








