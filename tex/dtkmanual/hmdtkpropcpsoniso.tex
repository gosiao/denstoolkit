                         Compilation date: Nov 19 2020

                                 Version: 1.2.0

                         :-) Created by: JMSA/JMHP (-:

********************************************************************************


Usage:

	dtkpropcpsoniso wf?name [option [value(s)]] ... [option [value(s)]]

Where wf?name is the input wfx(wfn) name, and options can be:
Where wf?name is the input wfx(wfn) name, and options can be:

  -a aG1 aG2	Sets the gnuplot angles to be aG1 and aG2.
            	  Use dtkqdmol to check and set these angles.
  -A aX aY aZ	Sets the view angles to be aX, aY, and aZ.
  -c cAtNum 	Sets the atom cAtNum to be the center around which the
            	  cap isosurface is computed.
  -H        	Hides the isosurface in png image (pov file).
  -k        	Keeps the pov-ray script (see also option P, below).
  -l dAtNum 	Sets the direction point to be the coordinates of atom
            	  dAtNum. See the scheme at the end of this help menu.
  -m a1 a2  	Sets the direction point to be the middle point located
            	  between the coordinates of atoms a1 and a2.
            	  See scheme at the end of this menu.
  -C a1 a2 a3	Sets the direction point to be the centroid of the
            	  triangle formed by the coordinates of atoms
            	  a1, a2, and a3.
            	  See scheme at the end of this menu.
  -p prop   	Sets the property evaluated at the isosurface to be
            	  prop, which is a char that can be any of the listed
            	  fields enumerated below for option -I.
  -I prop v 	Sets the field to compute the isosurface to be prop,
            	  and the isosurface value to be v. prop is a char,
            	  which can be (this is valid for options -I and -p):
         		d (Density)
         		g (Magnitude of the Gradient of the Density)
         		l (Laplacian of density)
         		K (Kinetic Energy Density K)
         		G (Kinetic Energy Density G)
         		E (Electron Localization Function -ELF-)
         		L (Localized Orbital Locator -LOL-)
         		M (Magnitude of the Gradient of LOL)
         		P (Magnitude of the Localized Electrons Detector -LED-)
         		r (Region of Slower Electrons -RoSE-)
         		s (Reduced Density Gradient -s-)
         		S (Shannon Entropy Density)
         		V (Molecular Electrostatic Potential)
  -o basename	Set the base name for the output files to be basename.
            	  (If not given the program will use the wf?name as the
            	  the base name.)
  -P        	Generates a pov-ray script and renders it. Notice: this requires
            	   povray to be installed in your system.
  -r rlev   	Refines the isosurface cap mesh. rlev is a number used to
            	  perform a series of subdivisions of an initial icosahedron 
            	  (rlev=4 by default). Each iteration increases
            	  the number of triangles by a factor of 4, and it must be >0.
            	  rlev>8 is not recommended as it may cause numerical issues.
  -s        	Sets spacefilling mode. This mode activates the spacefilling
            	  view of the atoms (uses VdW atomic radius to draw the atoms).
  -t        	Draw transparent isosurface.
  -v vrbslev	Sets the verbose level to be vrbslev. This prints information according to
            	  the level chosen. Default: vrbslev=0.
  -V     	Displays the version of this program.
  -h     	Display the help menu.

  --use-cube cub 		Uses the cube file 'cub' to generate the isosurface,
                 		  as opposed to generate a cap around a selected atom
                 		  and direction. This option overrides all options
                 		  related to the cap. E.g. the mesh cannot be refined,
                 		  because the mesh is determined from the cube sampling;
                 		  
  --help    		Same as -h
  --version 		Same as -V
********************************************************************************
--------------------------------------------------------------------------------
Scheme of the cap drawn when the isosurface is option a).
(See description of the program following DTK logo, above.)
Below, r(c) is the radius vector
of the center atom (given through option -c), and r(d)
is the radius vector of the direction point (given through options
-l, -m, or -C).
The vector r(c)-r(d) is used to determine the orientation of the
isosurface cap, which is located in the opposite direction of r(d),
relative to r(c), i.e. the cap points in the same direction as
r(c)-r(d):
--------------------------------------------------------------------------------
                                     _
                                       -
                                         \
                                           \
                                            |
       o  -------[r(c)-r(d)]------> o        |<---Isosurface CAP
       |                            |       |
       |                   r(c)_____|      /
       |                                 /
     r(d)                            _ -

--------------------------------------------------------------------------------
********************************************************************************
