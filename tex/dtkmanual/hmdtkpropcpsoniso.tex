                         Compilation date: Nov  1 2024

                                 Version: 1.6.3

                         :-) Created by: JMSA/JMHP (-:

********************************************************************************


Usage:

	dtkpropcpsoniso wf?name [option [value(s)]] ... [option [value(s)]]

Where wf?name is the input wfx(wfn) name, and options can be:
Where wf?name is the input wfx(wfn) name, and options can be:

  -a aG1 aG2	Sets the gnuplot angles to be aG1 and aG2.
            	  Use dtkqdmol to check and set these angles.
  -A aX aY aZ	Sets the view angles to be aX, aY, and aZ.
  -c cAtNum 	Sets the atom cAtNum to be the center around which the
            	  cap isosurface is computed.
  -d ang    	Set maximum degree aperture around the direction of the cap
            	  to be ang. Default: d=75.
  -H        	Hides the isosurface in png image (pov file).
  -J        	Setup alpha- and beta-spin density matrices.
  -k        	Keeps the pov-ray script (see also option P, below).
  -l dAtNum 	Sets the direction point to be the coordinates of atom
            	  dAtNum. See the scheme at the end of this help menu.
  -m a1 a2  	Sets the direction point to be the middle point located
            	  between the coordinates of atoms a1 and a2.
            	  See scheme at the end of this menu.
  -C a1 a2 a3	Sets the direction point to be the centroid of the
            	  triangle formed by the coordinates of atoms
            	  a1, a2, and a3.
            	  See scheme at the end of this menu.
  -p prop   	Sets the property evaluated at the isosurface to be
            	  prop, which is a char that can be any of the listed
            	  fields enumerated below for option -I.
            	  (This option is included for future implementations,
            	  and in this version, only MEP[V] is implemented).
  -I prop v 	Sets the field to compute the isosurface to be prop,
            	  and the isosurface value to be v. prop is a char,
            	  which can be (this is valid for options -I and -p)
            	  (This option is included for future implementations,
            	  and in this version, only rho[d] is implemented,
            	  and by default the isovalue is v=0.001.)
         		d (Electron Density --Rho--)
         		g (Magnitude of the Gradient of the Electron Density)
         		l (Laplacian of Electron Density)
         		K (Kinetic Energy Density K)
         		G (Kinetic Energy Density G)
         		e (Ellipticity)
         		E (Electron Localization Function --ELF--)
         		L (Localized Orbital Locator --LOL--)
         		M (Magnitude of the Gradrient of LOL)
         		P (Magnitude of Localized Electrons Detector)
         		r (Region of Slow Electrons --RoSE--)
         		s (Reduced Density Gradient --s--)
         		S (Shannon-Entropy Density)
         		V (Molecular Electrostatic Potential)
         		v (Virial Potential Energy Density)
         		D (Density Overlap Regions Indicator --DORI--)
         		b (Spin density)
         		q (One electron disequilibrium --Rho squared--)
         		u (Scalar Custom Field)
  -o basename	Set the base name for the output files to be basename.
            	  (If not given the program will use the wf?name as the
            	  the base name.)
  -P        	Generates a pov-ray script and renders it. Notice: this requires
            	   povray to be installed in your system.
  -r rlev   	Refines the isosurface cap mesh. rlev is a number used to
            	  perform a series of subdivisions of an initial icosahedron 
            	  (rlev=4 by default). Each iteration increases
            	  the number of triangles by a factor of 4, and it must be >0.
            	  rlev>8 is not recommended as it may cause numerical issues.
  -s        	Sets spacefilling mode. This mode activates the spacefilling
            	  view of the atoms (uses VdW atomic radius to draw the atoms).
  -t        	Draw transparent isosurface.
  -v vrbslev	Sets the verbose level to be vrbslev. This prints information according to
            	  the level chosen. Default: vrbslev=0.
  -V     	Displays the version of this program.
  -h     	Display the help menu.

  --use-cube cub 	Uses the cube file 'cub' to generate the isosurface,
                 	  as opposed to generate a cap around a selected atom
                 	  and direction. This option overrides all options
                 	  related to the cap. E.g. the mesh cannot be refined,
                 	  because the mesh is determined from the cube sampling;
                 	  
  --pka-carbac   	Assume the input molecule is a carboxylic acid and estimate
                 	  its pKa. This option overrides options -I, -p, and -r.
  --pkb-prim-amine	Assume the input molecule is a primary amine and estimate
                 	  its pKb. This option overrides options -I, -p, and -r.
  --pkb-sec-amine	Assume the input molecule is a secondary amine and estimate
                 	  its pKb. This option overrides options -I, -p, and -r.
  --pkb-ter-amine	Assume the input molecule is a tertiary amine and estimate
                 	  its pKb. This option overrides options -I, -p, and -r.
  --help    		Same as -h
  --version 		Same as -V
********************************************************************************
--------------------------------------------------------------------------------
Scheme of the cap drawn when the isosurface is option a).
(See description of the program following DTK logo, above.)
Below, r(c) is the radius vector
of the center atom (given through option -c), and r(d)
is the radius vector of the direction point (given through options
-l, -m, or -C).
The vector r(c)-r(d) is used to determine the orientation of the
isosurface cap, which is located in the opposite direction of r(d),
relative to r(c), i.e. the cap points in the same direction as
r(c)-r(d).
Example1: if the molecule is a carboxylic acid, the atom c is the -COOH
  hydrogen, and atom d is the -OH oxygen (of the -COOH group).
Example2: if the molecule is CH3Cl and you wish to study the exterior
  cap around the Cl, then the atom c is Cl and atom d is C.
Example3: If the molecule is CH3NH2 and you wish to analize the
  nitrogen electron pair, then the atom c is N, and the cap can
  be oriented through option '-C ha hb ca', where ha and hb are
  the numbers that identify the -NH2 hydrogens and ca is the number
  that identify C (according to the atoms order of appearance in
  the wfx/wfn file).
--------------------------------------------------------------------------------
                                     _
                                       -
                                         \
                                           \
                                            |
       o  -------[r(c)-r(d)]------> o        |<---Isosurface CAP
       |                            |       |
       |                   r(c)_____|      /
       |                                 /
     r(d)                            _ -

--------------------------------------------------------------------------------
********************************************************************************
