Usage:

	dtkplane wf?name [option [value(s)]] ... [option [value(s)]]

Where wf?name is the input wfx(wfn) name, and options can be:
Where wf?name is the input wfx(wfn) name, and options can be:

  -a a1 a2 a3	Define the atoms  (a1,a2,a3) used to define the plane.
            	  If this option is not activated, the program will 
            	  choose a default plane, but you may not like the view.
            	  Note: if the *.wfn (*.wfx) file has only one or two atoms
            	  this option must not be used. The program will define
            	  a plane which includes that(those) one(two) atom(s).
  -n  dim   	Set the number of points for the tsv file per direction.
            	  The tsv will have the dimensions dim x dim.
  -o outname	Set the output file name.
            	  (If not given the program will create one out of
            	  the input name; if given, the tsv, gnp and pdf files will
            	  use this name as well --but different extension--).
  -p prop	Choose the property to be computed. prop is a character,
         	  which can be (d is the default value): 
         		d (Density)
         		g (Magnitude of the Gradient of the Density)
         		l (Laplacian of density)
         		K (Kinetic Energy Density K)
         		G (Kinetic Energy Density G)
         		E (Electron Localization Function -ELF-)
         		L (Localized Orbital Locator -LOL-)
         		M (Magnitude of the Gradient of LOL)
         		N (Gradient of LOL)
         		p (Localized Electrons Detector -LED-)
         		P (Magnitued of Localized Electrons Detector)
         		r (Region of Slow Electrons -RoSE-)
         		s (Reduced Density Gradient -s-)
         		S (Shannon Entropy Density)
         		u (Scalar Custom Field)
         		U (Vector Custom Field)
         		V (Molecular Electrostatic Potential)
         		v (Virial Potential Energy Density)
  -P     	Create a plot using gnuplot.
  -c     	Show contour lines in the plot.
  -k     	Keeps the *.gnp file to be used later by gnuplot.
  -l     	Show labels of atoms (those set in option -a) in the plot.
  -L     	Show the labels of ALL of the atoms in the wf? file.
  -v     	Verbose (display extra information, usually output from third-
         	  party sofware such as gnuplot, etc.)
  -z     	Compress the tsv file using gzip (which must be intalled
         	   in your system).
  -V        	Displays the version of this program.
  -h     	Display the help menu.

  --help    		Same as -h
  --version 		Same as -V
********************************************************************************
  Note that the following programs must be properly installed in your system:
                                    gnuplot
                                    epstool
                                    epstopdf
                                      gzip
********************************************************************************
