Usage:

	dtkdrawer inputmolecule [option [value(s)]] ... [option [value(s)]]

Where inputmolecule is the input file name. Valid input formats:
Where inputmolecule is the input file name. Valid input formats:
            	wfx (draw the molecule)
            	wfn (draw the molecule)
            	xyz (draw the molecule)
            	cpx (draw the molecule, critical points, and
            	      gradient paths)
Options can be:

  -a        	Draw transparent spheres around each nucleus.
  -C        	Draw bonds as cylinders.
  -c  cpt   	Select the Critical points to include (cpx files only;
            	  by default, all CPs are included). cpt is a series
            	  of letters that can be:
            		a: include ACPs
            		b: include BCPs
            		r: include RCPs
            		c: include CCPs

  -g  gpt   	Select the Gradient paths to include (cpx files only;
            	  by default, all GPs are included).
            	  gpt is a series of letters
            	  that can be:
            		b: include BGPs
            		r: include RGPs
            		c: include CGPs

  -o outname	Set the output file name.
            	  (If not given the program will create one out of
            	  the input name; if given, the gnp file and the pdf will
            	  use this name as well --but different extension--).
  -O a b c  	Orient the molecule using three atoms (indices: a, b, c
            	 according to the molecule's input geometry). The order is
            	  important, as the align vectors are defined as follows
            	  (x, y, and z are, here, the vectors left-right,
            	  bottom-up, and screen-to-user, relative to the screen):
            	  x=(c-b), y=a-dotProduct(c,(b-a)), and z=crossproduct(x,y).
            	  The atoms will be placed in the screen plane, and they will
            	  look as in scheme 1, below.
  -s        	Set spacefilling mode. This mode activates the spacefilling
            	  view of the atoms (uses VdW atomic radius to draw the atoms).
  -T        	Draw the gradient paths as tubes (as opposed to as a series
            	  of small spheres.
  -v verbLev	Set the verbose level to be verboseLevel. The greater
            	  verbLev is, the greater the information displayed on the
            	  screen. verboseLevel=0 minimizes the information.
  -V        	Display the version of this program.
  -w width  	Set the width (pixels) of the png to be width (using multiples of
            	  600 is recommended.
  -x alpha  	Set the angle GNUPlotAngle1=alpha (in povray file).
  -y beta   	Set the angle YAngle=beta (in povray file).
  -z gamma  	Set the angle GNUPlotAngle2=gamma (in povray file).
  -Z zmfact 	Set the zoom factor to be zmfact. This will alter the camera
            	  position by a factor zmfact.
  -h     	Display the help menu.

  --help    		Same as -h
  --version 		Same as -V
  --no-png  		Preclude the png rendering; povray will not be called.
********************************************************************************
  Note that the following programs must be properly installed in your system:
                                    gnuplot
                                    epstopdf
                                      gzip
********************************************************************************
--------------------------------------------------------------------------------
            	           a
            	           |
            	          y|
            	           |--------c
            	           |________|______
            	           / b       x       
            	          /
            	       z / 
  Scheme 1: View of the aligned atoms (see option -O).
--------------------------------------------------------------------------------
